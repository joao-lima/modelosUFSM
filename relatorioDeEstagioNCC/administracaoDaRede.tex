\documentclass[a4paper,11pt]{article}
\usepackage[portuges]{babel}
\usepackage{times}
\usepackage{indentfirst}
\usepackage[T1]{fontenc}
\usepackage[utf8]{inputenc}
\usepackage{url}
\usepackage[numbers,sort]{natbib}

\title{
        {\small
        {Universidade Federal de Santa Maria} \hfill
        {Curso de Ciência da Computação} \\
        \vskip -4mm
        Centro de Tecnologia \hfill
        Núcleo de Ciência da Computação
        }\\~\\
        {Relatório de atividade}\\[4mm]
        {\large \sc Estágio: \textbf{Administração da Rede do NCC}}
        \\[4mm]
        {\large Período: \textbf{agosto/2005 - outubro/2006}}
}

\author{
        {Nome: \textbf{João Vicente Lima}}\ \
        {Matrícula: \textbf{2310144}}\\
        {Orientador: \textbf{Prof.\ Benhur de Oliveira Stein}}\\
}

\date{}

\begin{document}

\maketitle

\section{Introdução}

Este relatório tem por objetivo descrever de forma sucinta os trabalhos
desenvolvidos pelo estagiário, desde agosto de 2005, até a presente data,
outubro de 2006.

As próximas seções apresentam as principais atividades desenvolvidas e uma
breve descrição do trabalho desenvolvido em cada uma destas atividades.

\section{Manutenção dos Servidores da Rede}

Desde agosto de 2005 foram realizadas várias instalações e migrações
de serviços dos principais servidores da rede. No princípio, foram
feitos estudos encima do estado em que a rede se encontrava,
como identificação e entendimento dos principais serviços. Neste fase
foi possível identificar serviços essenciais da rede como o servidor
de nomes de domínio (DNS\footnote{Domain Name Server})
\cite{dnsr,dnshowto,dnsstuff,iscbind} e de nomes de grupos de
usuários (LDAP\footnote{Lightweight Directory Access Protocol})
\cite {openldap,ldapbrasil}.
Após este
período foi feito uma análise, junto com o orientador deste estágio, sobre
melhorias e atualização dos serviços existentes.

A medida que os servidores e serviços eram definidos, passávamos para
a fase de migração. Esta fase foi acelerada quando na segunda metade
deste ano um dos principais servidores teve problemas com disco. Assim
tivemos a necessidade de elimina-lo do quadro de servidores e passar
os serviços existentes deste para outros com uma carga menor de
trabalho.

Ocasionalmente, ocorriam casos em que peças dos mesmos precisavam ser
trocadas. Podemos citar como exemplo um dos servidores necessitou a
troca da fonte de alimentação no dia 30 de outubro,
devido a uma queda de luz que ocorreu no domingo, 29 de outubro.

\section{Manutenção do Sistema Integrado}

O NCC possui um gerenciamento centralizado dos serviços DNS, DHCP e
NFS\footnote{Network File System} \cite{nfshowto} de forma simples e
prática conforme \cite{sistemancc}, que utiliza \textit{shell scripts}
~\cite{advancedbash} e \textit{makefiles}~\cite{gnumake}. 
Este sistema depende da
localização dos serviços nos servidores para ter acesso aos arquivos
de configuração. Dessa forma, a cada migração de servidores e serviços
é necessária a manutenção deste sistema. Assim como sua utilização
para o cadastro e alteração dos computadores que compõem a rede.

\section{Manutenção do Gerenciamento e Cadastro de Usuários}

Num primeiro momento, foi preciso entender como a autenticação era
efetuada no NCC com LDAP, SAMBA \cite{samba} e \textit{email}.
Após foi realizado um estudo de como o
sistema de controle gerenciava estes usuários, como os criava e
alterava. Com estes dados foi realizado uma reescrita de algumas
partes deste sistema, para atender as necessidades atuais
e para facilitar futuras alterações, principalmente no que diz
respeito aos acessos ao serviço LDAP.

\section{Manutenção e Controle de Sistemas Operacionais}

Para entender o sistema de carregamento dos sistemas operacionais dos
laboratórios do NCC foi preciso estudar o \textit{Remote Boot}
~\cite{remoteboot} com o intuito de dar continuidade ao melhoramento
deste sistema. Todavia, ele mostrou-se muito limitado para as novas
versões de sistemas operacionais como \textit{Linux}~\cite{linuxkernel}
e o \textit{Windows}~\cite{windows}, pois não foi mantido com novas
versões e atualizações. Assim foi procurado alternativas simples que
proporcionem a instalação das últimas versões dos programas
disponíveis.

Uma delas baseou-se em ferramentas oferecidas pelo sistemas
\textit{Linux} com o SYSLINUX~\cite{syslinux} e o
NFSRoot~\cite{nfsroot}. Primeiramente o SYSLINUX oferece o
\textit{pxelinux} que é possível iniciar um computador através da rede
e mostrar um menu simples com algumas opções sobre o \textit{kernel}
do sistema operacional \textit{Linux} que se deseja carregar. Depois
da escolha usa-se o \textit{NFSRoot} que permite montar o sistema de
arquivos remotamente, sem a necessidade de acesso aos disco da
máquina. Com este sistema totalmente remoto, é possível manter um
mecanismo de recuperação dos computadores sem a intervenção direta dos
administradores.

Para tanto, primeiro criou-se uma instalação completa de um sistema
\textit{GNU/Linux} em um servidor. Logo em seguida, criou-se o sistema
carregado remotamente com \textit{shell scripts} que executam a tarefa
de instalação dos arquivos do sistema completo no disco local e
construir uma instalação completa automaticamente. As dificuldades
encontradas nesta contrução foram o estudo do SYSLINUX e sua forma
de funcionamento,
e também a criação de um sistema que executasse remotamente baseada em
uma instalação tradicional. Atualmente este sistema de atualização é
aplicado em cerca de nove computadores do NCC.

A outra solução, que foi proposta pelo orientador deste estágio, é a
utilização do LTSP\footnote{Linux Terminal Server Project}
~\cite{ltsp,ltspgentoo}
para a
criação de terminais remotos sem a necessidade de discos de
armazenamento. Esta alternativa surgiu de dois problemas enfrentados
que eram a falta de recursos para novos computadores, e as
dificuldades de manter sistemas independentes em cada um deles. Hoje
em dia, o LTSP está em testes no NCC e é utilizado por sete
computadores que estão sem discos.

\section{Novo Sistema de Quotas de Impressão}

Um  novo sistema de quotas de impressão foi instalado no NCC, chamado
de Pykota~\cite{pykota}. Através dele é possível ter controle
dos usuários que utilizam a impressora, assim como gerenciar o número
de folhas que cada um consome por documento impresso. Este sistema
também permite estabelecer os preços por folha impressa e armazena o
quanto cada usuário tem direito a imprimir. O Pykota oferece vários
aplicativos de controle de usuários, impressoras e quotas. No NCC
usamos o MySQL~\cite{mysql,mysqlgentoo} para o armazenamento dessas
informações, e \textit{shell scripts} para facilitar o controle dos
usuários no sistema.

\section{Virtualização de Serviços}

Ultimamente, uma forte tendência foi virtualizar serviços através do
XEN~\cite{xen,xengentoo} a fim de
permitir flexibilidade na administração, o que possibilita sistemas
específicos para os propósitos de cada serviço.  Esta tarefa foi a mais
complicada deste estágio, pois foi aplicado todos os conhecimentos
adquiridos com os trabalhos anteriores anteriores,
como configuração de serviços básicos,
configuração e alteração do núcleo do sistema XEN, entre outras.

Um exemplo é o servidor SSH\footnote{Security Shell
Client}~\cite{ssh} utilizado pelo NCC para acesso externo de seus
usuário a rede local. Este sistema deve ser bem reduzido, além de
manter a privacidade entre os usuários que o compartilham. Dessa forma,
o sistema virtual foi montado conforme as exigências sem
necessitar de um computador próprio para esta tarefa. Atualmente
tem-se outro sistema virtual, com DNS e DHCP, que executa
concorrentemente com o
anterior. Neste último é desnecessário ter as limitações que o serviço
de SSH externo requer. Assim ambos executam em ambientes distintos e
sem interferir um no outro.

Uma considerável vantagem de um sistema virtual é a possibilidade de
experimentos para aumento da segurança. Para o serviço externo SSH,
criou-se um \textit{patch} para o \textit{kernel} XEN versão 2.6.16.28, que
esconde os processos de outros usuários e do sistema caso o usuário não
tenha permissões de administrador. Além disso, esta modificação elimina
qualquer característica que identifique o sistema em execução é
uma máquina virtual XEN, como
versão e informações contidas na pasta de sistema \textit{/proc}.
Este \textit{patch} foi baseado em parte pelo projeto
GRSecurity~\cite{grsec}.

\section{Verificação, Diagnóstico e Conserto de Máquinas}

Para a manutenção do bom funcionamento é necessário uma verificação
constante dos equipamentos.
Esta verificação consiste basicamente numa tarefa diária de 
inicialização e
visualização do funcionamento ou não de todos os computadores que fazem
parte dos laboratórios temáticos.

No caso de alguma máquina apresentar alguma anormalidade durante o 
processo 
de carga e inicialização do sistema é feito um diagnóstico para
identificar
o ponto de falha e sanar o problema.
Este processo muitas vezes leva a uma checagem dos componentes físicos da
máquina.
A máquina é aberta e são testados componentes como placa de rede, memória,
processador, disco, placa mãe e fonte de alimentação.
São testadas todas as possibilidades até que a origem da falha de operação
seja detectado.
Se for problema em algum dos dispositivos físicos da máquina, é feita a
troca do mesmo.

As falhas e consertos de dispositivos físicos das máquinas ocorrem com 
uma freqüência menor que a atualização e instalação de 
programas/aplicativos, por exemplo.
Contudo, é gasto um tempo considerável quando da ocorrência de problemas
deste gênero.

\section{Conclusão}

A tarefa de administração de uma rede de computadores requer uma grande 
disponibilidade de tempo e recursos humanos.
A automatização de muitas das tarefas atribuídas aos administradores
facilita e agiliza o processo.

Desde 2005, mesmo com alguma experiência em ambientes
\textit{GNU/Linux}, foram incansáveis horas de aprendizado e trabalho.
Mesmo com a falta de recursos e descaso com os equipamentos dos
administradores do NCC, foi possível ter uma experiência fenomenal.
Sem contar a assistência a usuários do laboratório que muitas vezes
tinham pouco convívio com ambientes de rede e com a forma de
utilização dos recursos do NCC.

Este relatório teve como objetivo apresentar brevemente as principais
atividades desenvolvidas pelo estagiário durante o período referido.

\bibliographystyle{plainnat}
\bibliography{referencias}

\end{document}
